\documentclass[a4wide,12pt]{scrartcl}
\usepackage{fancyhdr}
\usepackage{alltt}
\usepackage[T1]{fontenc}
\usepackage[utf8]{inputenc}
\usepackage[french]{babel}
\usepackage{xspace}
\usepackage{hyperref}

\selectlanguage{french}
\setlength{\parskip}{2ex plus 1ex minus 1ex}
\widowpenalty=10000
\clubpenalty=10000

\newcommand{\qui}{PromoLily.ORG\xspace}
\newcommand{\quand}{\today}

\hypersetup{pdftex, colorlinks=true, linkcolor=blue, citecolor=blue, filecolor=blue, urlcolor=blue, pdftitle=Règlement intérieur de \qui au \quand, pdfauthor=, pdfsubject=\qui, pdfkeywords=}

\renewcommand{\thesection}{Article~\arabic{section}}
\renewcommand{\thesubsection}{\arabic{section}\alph{subsection}}

\pagestyle{fancyplain}
\lhead{Règlement intérieur de l'association \qui}
\chead{}
\rhead{\quand}

\begin{document}

\section*{Préambule}
Ceci constitue le règlement intérieur de l'association \qui.\\
Conformément aux statuts de l'association, il est établi et voté par le
conseil d'administration.

Ce règlement intérieur a été approuvé lors de la réunion du conseil
d'administration du \quand. 

Tout adhérent s'engage au vu des statuts et du règlement intérieur de
l'association.


\section{Modalités de réalisation des objectifs}

L'association \qui, outre les objectifs définis à l'article 2a
de ses statuts, entend agir dans les domaines suivants~:

\begin{itemize}
\item Mise en commun des connaissances et compétences dans le domaine
  de la traduction française, de l'adaptation française d'interfaces
  et de la création de documentations françaises, en particulier dans
  le domaine des systèmes et logiciels libres et des standards ouverts
  en général, de tous les membres de l'association. 
\item Assistance aux membres de l'association, mais aussi, point
  incontournable, à tout utilisateur de LilyPond qui en ferait la
  demande, dans le domaine de l'adaptation au français, de
  l'internationalisation, de la traduction en français et de la
  rédaction de documentations françaises.
\item Regroupement régulier des membres de l'association, dans un but
  d'échange et de cohésion~; les membres ou correspondants éloignés de
  l'association peuvent être dispensés de présence régulière.
\item Participation à des manifestations telles que des foires, des
  salons, des journées des associations et tout autre événement où la
  promotion des systèmes et logiciels libres est possible.
  Organisation d'événements ou interventions à des fins de promotion
  des systèmes et logiciels libre en français et de l'adaptation des
  logiciels à la langue française.
\item Participation à des projets de plus grande envergure, dans le
  cadre des objectifs de l'association, et avec des organisations
  extérieures aux objectifs similaires ou compatibles.
\end{itemize}

Cette liste n'est évidemment pas limitative, et toute action conforme
aux statuts et non listée ci-dessus pourra cependant être entreprise
après délibération et accord du conseil d'administration. 


\section{Cotisations}
\subsubsection{Membres actifs}
Le montant de la cotisation des membres actifs est fixé pour l'année
de référence (voir l'article~3 de ce règlement intérieur) à la somme
de 15~euros.

Toute cotisation versée à l'association est définitivement acquise. Il
ne saurait être exigé un remboursement de cotisation en cours d'année
en cas de démission, d'exclusion, ou de décès d'un membre.

\section{Définition de l'année en cours}
L'année de fonctionnement de l'association et donc de cotisation est
fixée du 1\ier{}~juillet au 30~juin de l'année civile suivante. 


\section{Conditions d'admission des membres}
Le conseil d'administration se réserve le droit de dispenser une
personne physique de cotisation. 

La liste des membres est publique pour l'ensemble des membres de
l'association sans distinction de la qualité de membre, à l'exception
des membres d'honneur.

Conformément à la loi n\textsuperscript{o}~78-17 du 6~janvier~1978
relative à l'informatique, aux fichiers et aux libertés, l'adhérent
dispose d'un droit d'accès et de rectification des données le
concernant auprès du secrétaire de l'association.

L'adhérent s'engage à porter à la connaissance de l'association
\qui toute modification portant sur son adresse postale, adresse
électronique, téléphone.


\section{Courrier électronique authentifié}
La définition d'un courrier électronique authentifié est subordonnée
aux possibilités de chiffrement et d'authentification disponibles, et
notamment de la validité de la signature électronique. Dans l'état
actuel des choses, aucun courrier électronique ne peut être considéré
comme authentifiable.


\section{Assemblées générales et réunions à distance}
L'association \qui se réserve le droit de considérer comme valable la
participation à une réunion du conseil d'administration ou du bureau
d'un membre empêché dès lors que celui-ci a fait parvenir à au moins
un des membres du bureau un courrier précisant sa position sur au
moins la moitié des points portés à l'ordre du jour.

L'association \qui se réserve en outre le droit de considérer que
l'usage d'un forum privé situé sur son site internet, d'un canal IRC
privé et réservé à ses membres puisse être assimilé à une réunion
ayant même valeur qu'une assemblée générale, sous réserve que l'ordre
du jour de cette réunion virtuelle ait été clairement annoncée sur la
liste de diffusion ou sur le canal IRC réservé aux membres, et que les
conditions de quorum d'une assemblée générale soient respectées. 

Le rapport moral, le rapport financier et le budget prévisionnel,
ainsi que toute information se rapportant à l'ordre du jour, pourront
être consultés sur la liste de diffusion réservée aux membres quinze
jours avant l'assemblée générale.


\section{Modalités d'élection}
Le droit de vote est réservé aux membres actifs, à jour de cotisation,
et faisant partie de l'association depuis au moins trois mois jour
pour jour à la date effective du vote. Est éligible tout membre actif
satisfaisant aux mêmes conditions.

Le conseil d'administration se réserve cependant le droit de réduire
cette durée à titre exceptionnel. Cette décision devra être dûment
motivée lors de l'assemblée générale.

Aucune condition de nationalité n'est requise, ni pour l'adhésion, ni
pour l'élection au conseil d'administration. Conformément à la
législation en vigueur, seuls les membres de nationalité française
peuvent être élus au bureau.


\section{Déclaration de candidature}
Les candidatures au conseil d'administration doivent être adressées au
bureau au moins sept jours avant la date de l'assemblée générale.
Celles-ci devront être accompagnées d'une présentation du candidat et
de sa profession de foi. 


\section{Vote par correspondance}
Le vote par correspondance est admis pour les assemblées générales. Le
bulletin de vote sera disponible sur le forum des membres. Il devra
être adressé dûment rempli sous double enveloppe au secrétaire de
l'association \qui au plus tard trois jours avant la réunion.
Les enveloppes seront ouvertes pendant le vote de l'assemblée
générale. Nous étudions la possibilité de mettre en place un vote par
internet anonyme et sécurisé.


\section{Acquisitions, aliénations ou locations immobilières}
Pour l'instant le règlement intérieur ne prévoit pas de règles plus
contraignantes que celles des statuts.


\section{Remboursement des dépenses}
En ce qui concerne les frais de déplacement, ceux-ci auront lieu sur
présentation du titre de transport ou de la facture des frais engagés
(carburant et péages). Les membres sollicitant le remboursement de
leurs frais de carburant s'engagent à faire en sorte que cette facture
soit représentative de leurs dépenses effectives, par exemple en
faisant le plein au début et à la fin du voyage et en fournissant la
seconde facture.


\section{Commissaires aux comptes}
L'assemblée générale peut décider de désigner un ou plusieurs
commissaires aux comptes si le budget prévisionnel présenté par le
trésorier le nécessite. Ces commissaires aux comptes seront élus parmi
les membres éligibles de l'association.

\end{document}
